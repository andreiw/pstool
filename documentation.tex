\documentclass[12pt]{article}
\begin{document}
\title{\texttt{pstool} documentation}
\author{Andrey Warkentin}
\maketitle
  
\section{What is \texttt{pstool?}}
\texttt{Pstool} computes a one-sided power spectrum of a signal - id est the
portion of a signal's power falling within given frequency bins, where
the signal is represented by the input data. The format of the input
data will be discussed in the ``\emph{Basic use}''
section. \texttt{pstool} computes the raw magnitudes squared (properly
scaled to satisfy Parseval's Theorem) by processing the results of a
one-dimensional real-to-complex FFT routine running in parallel on an 
MPI\footnote{Message Passing Interface} cluster. Parallel processing 
of the RFFT algorithm results in performance increases (shorter
running times) given a large data set to be processed. 
\section{Building and installing \texttt{pstool}}
\texttt{pstool} has been developed in a Linux environment with the GNU
Compiler Collection, and although it should build and run under a SVR4 or BSD
environment, this has not been tested. \texttt{pstool} has been shown
to be architecture independent, running both on AMD-64/Linux,
IA-32/Linux and PowerPC/Linux. An MPI compiler, supporting
libraries and runtime will need to be 
installed on the computers where \texttt{pstool} is to be compiled and
executed. At the time of this writing, \texttt{pstool} is known to
successfully build and execute using MPICH\footnote{a freely available,
portable implementation of MPI}. \texttt{pstool} should work fine with
LAM/MPI and other MPI 
implementations, however this has not been tested. FFTW\footnote{a C subroutine
library for computing the discrete Fourier transform} libraries and
headers will need to be installed on the computers where
\texttt{pstool} is to be compiled and executed. \texttt{pstool} has
been developed using the 2.x version of FFTW, due to a lack of
parallel transforms in version 3.x. At the time of this writing, the
latest FFTW 2.x release is FFTW 2.1.5. To build \texttt{pstool},
please execute the \texttt{configure} script in the pstool source
directory, and follow the on-screen directions. Please run
\texttt{./configure -h} to see available configuration options. The
configuration script itself requires the GNU Bourne-Again SHell.
\section{Basic use}
To run the tool, some input data is needed. This data represents the
signal to be analyzed. Depending on how the FFTW library was compiled,
the input format changes. By default the FFTW library is compiled with
double precision, thus the data should be stored as a stream of
IEEE-754 double (64-bit) precision floating-point values. If the FFTW
library was compiled with single precision, then the data should be
stored as a stream of IEEE-754 single (32-bit) precision
floating-point values. The basic usage of \texttt{pstool} is - 
\\
\begin{center}
\texttt{pstool -i input\_file\_name -o output\_file\_name -s sample\_rate}
\end{center}
...where \texttt{input\_file\_name} is the name of the file containing
  the input data to be analyzed, \texttt{output\_file\_name} is the
  name of the file in which to store the calculated power spectrum,
  and \texttt{sample\_rate} is the rate (in Hz) at which the signal,
  represented by the input data, was sampled. The calculated power spectrum is
  stored in the output file as comma-separated values (CSV), and is in
  ready form for use in plotting with \texttt{gnuplot} or other tools.
\section{Advanced use}
\texttt{pstool} has several other options which may be of
use in varying contexts. The \texttt{-f rfft\_output\_file\_name}
option will result in 
\texttt{pstool} saving the results of the Fourier transform to the
file specified for future analysis. The \texttt{-m} option will cause
the FFTW library used by \texttt{pstool} to produce near-optimal RFFTW
performance\footnote{by default the FFTW library uses a 'reasonable'
  (for a RISC-style CPU with many registers)
  plan for calculating the transform} by calculating several
transforms, measuring their 
execution time and selecting the best one. By itself the option has
few uses and will result in longer execution time for
\texttt{pstool}. However if many power spectra are to be found, then
coupling this option with the FFTW 
\emph{wisdom mechanism} will result in shorter execution times for
\texttt{pstool} in future program runs. The \texttt{-e} and
\texttt{-i} give access to the 
FFTW \emph{wisdom mechanism}. The \emph{wisdom 
mechanism} provides a way of re-using the 'knowledge' created by the
FFTW library, during it's computation of a transform, in future
program runs. The \texttt{-e wisdom\_output\_file\_name} option will
cause \texttt{pstool} to save the FFTW wisdom information created during
execution of \texttt{pstool} with the \texttt{-m} option. The
\texttt{-i wisdom\_input\_file\_name} option will 
cause \texttt{pstool} to import FFTW wisdom information, resulting in
a likelier shorter execution time. Although the use of the
\emph{wisdom mechanism} will result in shorter execution times for
\texttt{pstool}, there are many caveats associated with using it. The
wisdom information generated by the FFTW library depends heavily upon
the hardware/system (CPU, memory, virtual memory) configuration. It is
perfectly 
safe to use wisdom information generated during the run
of \texttt{pstool} on  (for example) an IBM RS/6000, in a run of
\texttt{pstool} on an 
AMD-64 cluster. However doing so will result in \emph{decreased}
performance. The general rule is that existing wisdom information
should not be re-used whenever there occurs a change of environment in
which it is used. This \emph{does} include recompiles of
\texttt{pstool} or of the FFTW libraries \texttt{pstool} depends on. 
\end{document}